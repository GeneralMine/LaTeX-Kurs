\section{Dokumentheader}
Im \textit{Header} werden einmalig globale Einstellungen getroffen, um unteranderem einheitliches Design zu garantieren oder Packete zu konfigurieren.
Eingeleitet wird der \textit{Header} durch \textit{documentclass} und geschlossen durch \textit{begin(document)}

\subsection{Dokumentklassen}
Dokumentklassen legen erste allgemeine Layoutregeln fest wie beispielsweise Ränder, Abstände, Schriftgrößen etc.
\subsubsection{Typische Klassen}
Die typischen Klassen nach Relevanz und Häufigkeit sortiert. Meist wird \textit{article} verwendet.
\begin{table}[h]
    \centering
    \begin{tabular}{rl}
        article & (Kurze) Artikel          \\
        report  & Reporte, Tagungsberichte \\
        book    & Bücher                   \\
        letter  & Briefe                   \\
        minimal & Minimalbeispiele         \\
        beamer  & Präsentationen
    \end{tabular}
    \caption{Die typischen Dokumentklassen}
    \label{typischeDokumentKlassen}
\end{table}

\subsubsection{KOMA-Klassen}
KOMA-Klassen sind moderne Erweiterungen der typischen Klassen. Sie modernisieren nicht nur das Layout und Design, sondern erlauben es einfach KOMA-Skripte zunutzen, wodurch mehr Funktionalität und Packete unterstützt wird.

\begin{table}[h]
    \centering
    \begin{tabular}{rl}
        scrartcl & Erweiterung von article   \\
        scrreprt & Erweiterung von report    \\
        scrbook  & Erweiterung von book      \\
        scrlttr2 & Sehr mächtige Briefklasse
    \end{tabular}
    \caption{Zusätzliche KOMA-Klassen}
    \label{KOMAKlassen}
\end{table}

\subsubsection{Wichtige \texttt{documentclass} Optionen}
Benutzung: \begin{verbatim}\documentclass[opt,opt,...]{Klasse}\end{verbatim}
\begin{table}[h]
    \centering
    \begin{tabular}{rl}
        titlepage           & Fügt Titelseite an Anfang ohne Seite mitzuzählen \\
        twocolumn           & Zweispaltiges Dokument                           \\
        twoside             & Seitenränder für Doppelseiten                    \\
        landscape           & Querformat                                       \\
        parskip             & Freizeile statt Einzug                           \\
        10pt / 11pt / 12pt  & Schriftgröße                                     \\
        letterpaper/a4paper & Papierformat
    \end{tabular}
    \caption{Wichtige documentclass Optionen}
    \label{documentclassOptionen}
\end{table}

\subsection{Titel}

\subsection{Wichtige Pakete}

\newpage