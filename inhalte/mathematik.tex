% Umgebungen und Mathe Sektion
\section{Mathe}

Im \textit{inline} Mathematik-Modus definieren wir die Masse $m$, Energie $E$ sowie die Lichtgeschwindigkeit \(c\), die wir anschließend im \textit{display} Mathematik-Modus einer Formel verwenden

\[
    E=mc^2
\]

Auch Variablen und griechische Symbole sind möglich
\[
    \alpha = \theta \cdot \gamma
\]

Brüche können mit \textit{frac} beschrieben werden
\[
    \alpha = \frac{\beta\cdot\alpha}{\Gamma}
\]

Im Gegensatz zum einfachen \textit{display} Mathematik-Modus sind Gleichungen in der \textit{equation} Umgebung nummeriert.

\begin{equation}
    \int_{0}^{\infty} f(x)\,\mathrm{d}x
\end{equation}

Mit \textit{align} kann man Formeln schön untereinander alignen.

\begin{align}
    a     & = b + c \\
    c + d & = e + f
\end{align}



\subsection{Maxwellgleichung}

In cgs-Einheiten und differentieller Form lauten die vier Maxwellgleichungen:

\begin{align}
    \nabla \cdot \vec{E}  & = 4\pi \rho                                     & \text{Gaußsches Gesetz}                        \\
    \nabla \cdot \vec{B}  & = 0                                             &                                                \\
    \nabla \times \vec{E} & = -\partial_{ct} \vec{B}                        & \text{Faradaysches Induktionsgesetz}           \\
    \nabla \times \vec{B} & = \frac{4\pi}{c}\vec{j} + \partial_{ct} \vec{E} & \text{Ampêre-Maxwellsches Durchflutungsgesetz}
\end{align}




\newpage